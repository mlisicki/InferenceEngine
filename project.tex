\documentclass[10pt,a4paper]{report}
\usepackage[utf8]{inputenc}
\usepackage{graphicx}

\begin{document}
\begin{LARGE}
\begin{center}
Inference engine project
\end{center}
\end{LARGE} 

\section{CSP}

state set - each state is defined by assignment to variables
action means transition between state

ok. so we need to define such problem for each concept

e.g. with rubik cube vars should be described segments
cons are defined in a graph for given concept (we check every concept!)

firs normalize segments, normalize constraints

start solving

check if var is in the domain

	it could be like i give the segments in var and they must fit approximately (within some defined bounds) those defined in csp graph( e.g. graph([domainset (dom_deg(seg1, point(100<X<105,195<Y<205), point(...) ), dom_seg(...))], [arc(dom_seg1,dom_seg2, connected), arc(...)])

those 'inside contraints' could be generated by some threshold or through learning

this would be matching - if is in the domain - assign id

then test constraints(?)

graph([domainset(seg1,seg2,seg3,....),[arc(seg1,seg2, connected),....])

ok. so domain would be only like what kind of values can have a segment (all values) to be considered a segment.

what of there are only nine segments meeting criteria

in the paper - each variable can take values from a finite domain

\end{document}
